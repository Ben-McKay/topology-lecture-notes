\documentclass{willowtreebook}

\Title{Topology Lecture Notes}
\Author{\texorpdfstring{Benjamin \scotsMc{}Kay}{Benjamin McKay}}
\BibliographyFile{topology-lecture-notes}

\DeclareFontFamily{OMX}{lmex}{}
\DeclareFontShape{OMX}{lmex}{m}{n}{<-> lmex10}{}

\DeclareUnicodeCharacter{2227}{\ensuremath{\wedge}}
\DeclareUnicodeCharacter{2207}{\ensuremath{\nabla}} %∇
\DeclareUnicodeCharacter{00A3}{\ensuremath{\LieDer}} %£
\DeclareUnicodeCharacter{02E9}{\ensuremath{\hook}} %˩
\DeclareUnicodeCharacter{2308}{\ensuremath{\lceil}}
\DeclareUnicodeCharacter{2309}{\ensuremath{\rceil}}
\DeclareUnicodeCharacter{230A}{\ensuremath{\lfloor}}
\DeclareUnicodeCharacter{230B}{\ensuremath{\rfloor}}
\DeclareUnicodeCharacter{03F5}{\ensuremath{\varepsilon}}
\DeclareUnicodeCharacter{2282}{\ensuremath{\subset}}%⊂
\DeclareUnicodeCharacter{03C6}{\ensuremath{\phi}}%ϕ
\DeclareUnicodeCharacter{03D5}{\ensuremath{\varphi}}%ϕ
\DeclareUnicodeCharacter{2297}{\otimes}
\DeclareUnicodeCharacter{03C9}{\ensuremath{\omega}}%
\DeclareUnicodeCharacter{222B}{\int}%
\DeclareUnicodeCharacter{221E}{\infty}%
\DeclareUnicodeCharacter{2211}{\sum}%
\DeclareUnicodeCharacter{03B1}{\alpha}%
\DeclareUnicodeCharacter{03B2}{\beta}%
\newunicodechar{ℝ}{\mathbb{R}}
\newunicodechar{ϕ}{\varphi}
\newunicodechar{⟨}{\left<}
\newunicodechar{⟩}{\right>}

%\NewDocumentCommand\textprime{}{\ensuremath{'}}

\usepackage{tensor}% for upper and lower indices

\usepackage{mathrsfs}%the \mathscr command for script fonts

\usepackage{mleftright}% fixes problems with \left and \right

\usepackage{standalone}

\usepackage{verbatim}
% For verbatim quotation of programming code.

\usepackage{siunitx}

% TiKZ graphics packages
\usepackage{brunnian}
\usepackage{tikz}
\usetikzlibrary{%
matrix,%
arrows,%
arrows.meta,%
calc,%
decorations.pathmorphing,%
decorations.pathreplacing,%
decorations.markings,%
decorations.fractals,%
fadings,%
hobby,%
knots,%
shadows,%
lindenmayersystems,%
shadings,%
backgrounds,%
angles,%
quotes}
\usepackage{pgfplots}
\usepgfplotslibrary{polar}
\usepgfplotslibrary{fillbetween}
\usepackage{tikz-3dplot}
%\usepackage{tkz-graph}
\usepackage{tikz-cd}
\usepackage{tqft}
\usepackage{cprotect}
%\usepackage[final]{sagetex}
\usepackage{dynkin-diagrams}
\usepackage{colortbl}

% We don't need to worry about the pdf page groups, so we ignore these warnings.
\pdfsuppresswarningpagegroup=1

%%.....Mathematics Commands
\def\cprime{\('\)} % For Russian names

% for definitions, A \defeq B means A is defined to be B.
\newcommand*{\defeq}{\mathrel{\vcenter{\baselineskip0.5ex \lineskiplimit0pt
                     \hbox{\scriptsize.}\hbox{\scriptsize.}}}%
                     =}


% I don't like the default \Re and \Im 
% for complex numbers.
\renewcommand{\Re}{\ensuremath{\operatorname{Re}}}
\renewcommand{\Im}{\ensuremath{\operatorname{Im}}}

% Various sets
\newcommand*{\Z}[1]{\ensuremath{\mathbb{Z}^{#1}}}
\newcommand*{\R}[1]{\ensuremath{\mathbb{R}^{#1}}}
\newcommand*{\Q}[1]{\ensuremath{\mathbb{Q}^{#1}}}
\newcommand*{\C}[1]{\ensuremath{\mathbb{C}^{#1}}}
\newcommand*{\Quat}[1]{\ensuremath{\mathbb{H}^{#1}}}

% Brackets
\newcommand*{\pr}[1]{\ensuremath{\left(#1\right)}}
\newcommand*{\curly}[1]{\ensuremath{\left\{#1\right\}}}
\newcommand*{\of}[1]{\ensuremath{\pr{#1}}}

% Matrices
\newcommand*{\matrixExtraVerticalSpace}{1.2}
\makeatletter
\renewcommand*\env@matrix[1][\arraystretch]{%
  \edef\arraystretch{#1}%
  \hskip -\arraycolsep
  \let\@ifnextchar\new@ifnextchar
  \array{*\c@MaxMatrixCols c}}
\makeatother

% Vectors
\newcommand*\vectr[1]%
{%%
\begin{pmatrix}#1{}\end{pmatrix}
}%%

% ball
\NewDocumentCommand\ball{somm}%[metric space]{radius}{center}
{%
\IfValueTF{#2}%
{B_{#3}\of{#4,#2}}%
{%% else
\IfBooleanTF{#1}%
{B_{#3}\of{#4}}%
{B_{#3}#4}%
}%%
}%

% closed ball
\NewDocumentCommand\ballClosed{somm}%[metric space]{radius}{center}
{%
\IfValueTF{#2}%
{\bar{B}_{#3}\of{#4,#2}}%
{%% else
\IfBooleanTF{#1}%
{\bar{B}_{#3}\of{#4}}%
{\bar{B}_{#3}#4}%
}%%
}%

% Open subset of
\NewDocumentCommand\op{m}{\text{open } \subset #1}

% little o
\newcommand*{\littleo}[1]{\ensuremath{o\of{#1}}}

% volume:
% \vol{X} for Vol X, \vol*{X} for Vol(X).
\NewDocumentCommand\vol{sm}%
{%
\IfBooleanTF{#1}%
{\ensuremath{\operatorname{Vol}\of{#2}}}%
{\ensuremath{\operatorname{Vol}#2}}%
}%

% area
\NewDocumentCommand\area{sm}%
{%
\IfBooleanTF{#1}%
{\ensuremath{\operatorname{Area}\of{#2}}}%
{\ensuremath{\operatorname{Area}#2}}%
}%

% length
\NewDocumentCommand\length{sm}%
{%
\IfBooleanTF{#1}%
{\ensuremath{\operatorname{length}\of{#2}}}%
{\ensuremath{\operatorname{length}#2}}
}%

\NewDocumentCommand\conv{mm}%
{%
#1*#2%
}%


% Transpose
\newcommand*{\trans}[1]{{#1}^t}

% Inner product 
\newcommand*{\ip}[2]{\ensuremath{\left<#1,#2\right>}}

% Inner product 
\NewDocumentCommand\Span{om}{\ensuremath{\IfValueTF{#1}{\left<#1\right>_{#2}}{\left<#2\right>}}}

% Partial derivative
\newcommand*{\pd}[2]{\ensuremath{\frac{\partial #1}{\partial #2}}}

% Number of integer points
\NewDocumentCommand\intsin{sm}%
{%
\IfBooleanTF{#1}%
{\tensor[^{\#}]{(#2)}{}}%
{\tensor[^{\#}]{{#2}}{}}%
}%

% Hook
\NewDocumentCommand\hook{}{\ensuremath{\mathbin{\hbox{\vrule height1.4pt width4pt depth-1pt \vrule height4pt width0.4pt depth-1pt}}}}

% Lie derivative
\NewDocumentCommand\LieDer{}{\ensuremath{\EuScript L}}

% Adjoint
%\NewDocumentCommand\ad{m}{\ensuremath{#1^*}}
\DeclareMathOperator{\Kernel}{ker}
\DeclareMathOperator{\Image}{im}
\DeclareMathOperator{\rank}{rank}
\DeclareMathOperator{\tr}{tr} % trace
\DeclareMathOperator{\Pf}{Pf} % Pfaffian

\NewDocumentCommand\coface{om}{T^+_{\IfValueT{#1}{#1}{}}{#2}}
\NewDocumentCommand\conormal{om}{T^{\perp}_{\IfValueT{#1}{#1}{}}{#2}}
\NewDocumentCommand\cofac{om}{{#2}^+_{\IfValueT{#1}{#1}{}}}
\NewDocumentCommand\conorm{om}{{#2}^{\perp}_{\IfValueT{#1}{#1}{}}}

\NewDocumentCommand\cofacequo{om}{T^-_{\IfValueT{#1}{#1}{}}{#2}}
\NewDocumentCommand\cofacquo{om}{{#2}^-_{\IfValueT{#1}{#1}{}}}


% Classical groups
\NewDocumentCommand\Gp{mm}{\ensuremath{\textrm{#1}\of{#2}}}
\NewDocumentCommand\GL{m}{\Gp{GL}{#1}}
\NewDocumentCommand\SL{m}{\Gp{SL}{#1}}
\NewDocumentCommand\PSL{m}{\mathbb{P}\Gp{SL}{#1}}
\NewDocumentCommand\PGL{m}{\mathbb{P}\Gp{GL}{#1}}
\NewDocumentCommand\SO{m}{\Gp{SO}{#1}}
\NewDocumentCommand\Orth{m}{\Gp{O}{#1}}
\NewDocumentCommand\Un{m}{\Gp{U}{#1}}
\NewDocumentCommand\Symp{m}{\Gp{Sp}{#1}}
\NewDocumentCommand\Or{m}{\operatorname{O}\left({#1}\right)}
\NewDocumentCommand\SOp{m}{\operatorname{SO}^{+}\left({#1}\right)}
\NewDocumentCommand\PO{m}{\operatorname{\mathbb{P}O}\left({#1}\right)}
\NewDocumentCommand\LieSO{m}{\LieAlgebra{SO}{#1}}
\newcommand{\LieGL}[1]{\mathfrak{gl}\left({#1}\right)}
\newcommand{\LieSL}[1]{\mathfrak{sl}\left({#1}\right)}
\newcommand{\LieOr}[1]{\mathfrak{o}\left({#1}\right)}
\newcommand{\LieSymp}[1]{\mathfrak{sp}\left({#1}\right)}
\newcommand{\PSymp}[1]{\operatorname{\mathbb{P}Sp}\left({#1}\right)}
\newcommand{\SU}[1]{\operatorname{SU}\left({#1}\right)}
\newcommand{\LieSU}[1]{\mathfrak{su}\left({#1}\right)}
\newcommand{\PSU}[1]{\operatorname{\mathbb{P}SU}\left({#1}\right)}
\newcommand{\LieUn}[1]{\mathfrak{u}\left({#1}\right)}
\newcommand{\PUn}[1]{\operatorname{\mathbb{P}U}\left({#1}\right)}
\NewDocumentCommand\Ad{om}{\IfValueTF{#1}{\operatorname{Ad}_{#1}#2}{\operatorname{Ad}#2}}
\NewDocumentCommand\ad{om}{\IfValueTF{#1}{\operatorname{ad}_{#1}#2}{\operatorname{ad}#2}}

\NewDocumentCommand\LieAlgebra{mm}{\ensuremath{\mathfrak{\MakeLowercase{#1}}\of{#2}}}
\newcommand{\LieG}{\mathfrak{g}}
\newcommand{\LieH}{\mathfrak{h}}

\NewDocumentCommand\solvrad{m}{\sqrt{#1}}
\NewDocumentCommand\Levi{m}{{#1}^{\text{ss}}}

\NewDocumentCommand\CC{m}{\ensuremath{{#1}_{\C{}}}}


% Matrix norm
\NewDocumentCommand\norm{m}{\left\|#1\right\|}

%%%%
%%%% Graphics
%%%%

\newcommand*{\TwoDPlotColourOne}{blue!50}
\newcommand*{\TwoDPlotColourTwo}{olive!50!blue}
\newcommand*{\TwoDPlotColourThree}{olive!75!blue}
\newcommand*{\TwoDPlotColourFour}{olive!50}


% Miffy: drawing a little face.
\tikzstyle _grid=[gray!10,thin]
\NewDocumentCommand\miffy{}
{%
\filldraw[drop shadow, rounded corners,fill=brown!20,draw=brown] (-.6,-.6) rectangle (.6,.6); % background
    \draw (0,0) circle (.5); % boundary of face
    \fill (-0.05555cm,0) circle (.03cm); % left eye
    \fill (+.388889cm,0.05555cm) circle (.03cm); % right eye
    \draw (+.1666cm,-.1666cm) -- +(.103cm,-.0407cm);
    \draw (+.1666cm,-.21444cm) -- +(+.111cm,.0444cm);
}%
\NewDocumentCommand\gridInsideMiffysFace{}
% A grid to lie inside Miffy's face, so that
% we can make area comparisons.
{%
    \begin{scope}
    \draw[clip] (0,0) circle (.5); % boundary of face
    \draw[step=0.1,_grid] (-1,-1) grid (1,1);
    \end{scope}
}%
\NewDocumentCommand\nullMiffy{mmmm}%{a}{b}{c}{d}
% Matrix
% (a b)
% (c d)
{%
\draw%
(-.5cm*{sqrt(#1*#1+#2*#2)},-.5cm*{sqrt(#3*#3+#4*#4)})%
 --
(.5cm*{sqrt(#1*#1+#2*#2)},.5cm*{sqrt(#3*#3+#4*#4)});
}%
\NewDocumentCommand\distortedMiffy{mmmm}%{a}{b}{c}{d}
% Matrix
% (a b)
% (c d)
{%
\begin{scope}[cm={{#1},{#2},{#3},{#4},(0,0)}]%
\miffy%
\end{scope}
}%
\NewDocumentCommand\gridBehindMiffy{mmmm}%{a}{b}{c}{d}
% A grid to be drawn behind miffy,
% where miffy is distorted by the matrix
% (a b)
% (c d)
{%
    \draw[_grid,step=.5cm]
    (-.5cm*{ceil(sqrt({#1}*{#1}+{#2}*{#2}))},-.5cm*{ceil(sqrt({#3}*{#3}+{#4}*{#4}))})
    grid
    (.5cm*{ceil(sqrt({#1}*{#1}+{#2}*{#2}))},.5cm*{ceil(sqrt({#3}*{#3}+{#4}*{#4}))}) ;
}%


\tikzfading[name=fade left,
left color=transparent!70,
right color=transparent!0]

% mid-point rule
\pgfplotsset{
	compat=1.16,
    midpoint segments/.code={\pgfmathsetmacro\midpointsegments{#1}},
    midpoint segments=3,
    midpoint/.style args={#1:#2}{
        ybar interval,
        domain=#1+((#2-#1)/\midpointsegments)/2:#2+((#2-#1)/\midpointsegments)/2,
        samples=\midpointsegments+1,
        x filter/.code=\pgfmathparse{\pgfmathresult-((#2-#1)/\midpointsegments)/2}
    }
}


% right hand sums
\pgfplotsset{
    right segments/.code={\pgfmathsetmacro\rightsegments{#1}},
    right segments=3,
    right/.style args={#1:#2}{
        ybar interval,
        domain=#1+((#2-#1)/\rightsegments):#2+((#2-#1)/\rightsegments),
        samples=\rightsegments+1,
        x filter/.code=\pgfmathparse{\pgfmathresult-((#2-#1)/\rightsegments)}
    }
}

% left hand sums
\pgfplotsset{
    left segments/.code={\pgfmathsetmacro\leftsegments{#1}},
    left segments=3,
    left/.style args={#1:#2}{
        ybar interval,
        domain=#1:#2,
        samples=\leftsegments+1,
        x filter/.code=\pgfmathparse{\pgfmathresult}
       }
}



% Draw a vector field in the plane.

\newwrite\tempfile

\newcounter{vectorfieldcounter}
\setcounter{vectorfieldcounter}{0}


\newcommand*{\vectorfieldplot}[4]%%%
%% \vectorfield{2*x}{1+y}{2x}{1+y}
%% draws the vector field (2x,1+y) over -1 < x < 1, -1 < y < 1, with a title.
%% \vectorfield{2*x}{1+y}{}{}
%% draws the vector field (2x,1+y) over -1 < x < 1, -1 < y < 1, with no title.
{%%%%
\stepcounter{vectorfieldcounter}%
\IfFileExists{XXX\thevectorfieldcounter.pdf}%%
{%%
\def\empt{}
\def\temp{#3}
\ifx\temp\empt%%
\begin{tabular}{c}%%
\includegraphics[width=2cm]{XXX\thevectorfieldcounter.pdf}%%
\end{tabular}
\else%%
\begin{tabular}{c}%%
\((#3,#4)\) \\{}%% 
\includegraphics[width=2cm]{XXX\thevectorfieldcounter.pdf}%% 
\end{tabular}\fi
}%%
{\immediate\openout\tempfile=XXX\thevectorfieldcounter.asy%
\immediate\write\tempfile{settings.outformat="pdf";
settings.prc=false;
settings.render=16;
import graph;
pen vectorfieldcolour=.3*blue+.6*white+.1*black+.3bp;
size(2cm,0);
pair a=(-1,-1);
pair b=(1,1);
path vector(pair z) 
{
	real x=z.x;
	real y=z.y;
	return (0,0)--(#1,#2);
}
add(vectorfield(vector,a,b,10,vectorfieldcolour,arrow=Arrow(2.5pt)));
clip(box(a,b));}
\immediate\closeout\tempfile%
\immediate\write18{asy XXX\thevectorfieldcounter.asy}%
}%
}%%%%




% drawing a cut-away sphere:
\def\k{0.55191496}

\tikzset{
    sphere color/.store in=\spherecolor,
    sphere scale/.store in=\spherescale,
    sphere color=blue,
    sphere scale=1,
    sphere/.style={
        ultra thick,
        line join=round,
        draw=#1!75!black,
        ball color=#1,
    },
    sphere inside/.style={
        shading angle=180,
        sphere=#1!25!gray!75!black
    }
}

\newenvironment{sphere}[1][]
    {
        \begin{scope}[x=(0:1cm), y=(90:1cm), z=(260:0.25cm), #1]
            \path [sphere inside=\spherecolor, scale=\spherescale] 
            circle [radius=1];
    }
    {
        \path let \n1={cos 10}, \n2={sin 10} in [sphere=\spherecolor, scale=\spherescale, even odd rule, opacity=0.5]
        circle [radius=1] 
        % Rotate 10 degrees around the y and x axes
        [x={(\n1, \n2^2, \n2*\n1)},
         y={(0, \n1, \n2)}, 
         z={(-\n2, -\n1*\n2, \n1^2)}] (0,1,0) 
                .. controls ++( 0, 0,\k) and ++(0,\k, 0) .. (0, 0, 1)
                .. controls ++(\k, 0, 0) and ++(0, 0,\k) .. (1, 0, 0) 
                .. controls ++(0, \k, 0) and ++(\k,0, 0) .. (0, 1, 0);
        \end{scope}
    }

% For the coffee cup in the manifolds chapter.
\tikzfading[name=fade out, inner color=transparent!0, outer color=transparent!100]

\newcommand*{\linearApproxBox}[6]%
{%
\draw[thick,white,opacity=.5]
({#1-#3-#5},{#2-#4-#6}) --
({#1+#3-#5},{#2+#4-#6}) --
({#1+#3+#5},{#2+#4+#6}) --
({#1-#3+#5},{#2-#4+#6}) --
cycle;
}%

\NewDocumentCommand\contMaps{O{}mO{\R{}}}{\ensuremath{C^{#1}({#2},{#3})}}
\NewDocumentCommand\smoothMaps{mO{\R{}}}{\contMaps[\infty]{#1}[#2]}


\newcommand*{\pickcolor}[2]{\definecolor{currentcolor}{rgb}{#1,.5,#2}}
\newcommand*{\nForms}[2]{\ensuremath{\Omega^{#1}\of{#2}}}
\newcommand*{\topForms}[1]{\nForms{\operatorname{top}}{#1}}
\newcommand*{\coboundaries}[2]{\ensuremath{B^{#1}\of{#2}}}
\newcommand*{\cocycles}[2]{\ensuremath{Z^{#1}\of{#2}}}
\newcommand*{\cohomology}[2]{\ensuremath{H^{#1}\of{#2}}}
\newcommand*{\deRhamcohomology}[2]{\ensuremath{H_{\text{dR}}^{#1}\of{#2}}}
\newcommand*{\compactcohomology}[2]{\ensuremath{H_c^{#1}\of{#2}}}
\newcommand*{\boundaries}[2]{\ensuremath{B_{#1}\of{#2}}}
\newcommand*{\cycles}[2]{\ensuremath{Z_{#1}\of{#2}}}
\NewDocumentCommand\homology{O{}mm}{\ensuremath{H^{#1}_{#2}\of{#3}}}
\NewDocumentCommand\BMhomology{mm}{\ensuremath{H^{\text{BM}}_{#1}\of{#2}}}
%\newcommand*{\homology}[2]{\ensuremath{H_{#1}\of{#2}}}
\usepackage{xspace}
\newcommand*{\Cech}{\v{C}ech\xspace}
\newcommand*{\CD}{\Cech--de~Rham\xspace}
\newcommand*{\CDchains}[2]{\ensuremath{\check\Omega^{#1}\of{#2}}}
\newcommand*{\CDcoboundaries}[2]{\ensuremath{\check{B}^{#1}\of{#2}}}
\newcommand*{\CDcocycles}[2]{\ensuremath{\check{Z}^{#1}\of{#2}}}
\newcommand*{\CDcohomology}[2]{\ensuremath{\check{H}^{#1}\of{#2}}}
\newcommand*{\CDd}{d}
\newcommand*{\topcohomology}[1]{\cohomology{\operatorname{top}}{#1}}
\newcommand*{\topcompactcohomology}[1]{\compactcohomology{\operatorname{top}}{#1}}
\newcommand*{\relativecohomology}[3]{\ensuremath{H^{#1}\of{#2;#3}}}
\newcommand*{\testforms}[2]{\ensuremath{\mathscr{D}^{#1}\of{#2}}}
\newcommand*{\allforms}[2]{\ensuremath{\mathscr{E}^{#1}\of{#2}}}
\newcommand*{\currentsdim}[2]{\ensuremath{\mathscr{D}'{}^{#1}\of{#2}}}
\newcommand*{\currentsdeg}[2]{\ensuremath{\mathscr{D}'_{#1}\of{#2}}}
\newcommand*{\supp}[1]{\ensuremath{\operatorname{supp} #1}}
\newcommand*{\Cnorm}[2]{\ensuremath{\norm{#2}_{C^{#1}}}}
\NewDocumentCommand\Cont{mm}{\ensuremath{C^{#1}\of{#2}}}
\NewDocumentCommand\Smooth{m}{\ensuremath{\Cont{\infty}{#1}}}
\newcommand{{\bull}}{{\scriptscriptstyle{\bullet}}}
\newcommand*{\complex}[1]{\ensuremath{#1^\bull}}
\newcommand*{\flag}[1]{\ensuremath{#1_\bull}}
\newcommand*{\flagvariety}[1]{\ensuremath{\operatorname{Fl}_{#1}}}
\newcommand*{\partialflagvariety}[2]{\ensuremath{\operatorname{Fl}^{#1}_{#2}}}
\newcommand*{\cohomologycomplex}[1]{\ensuremath{\cohomology{\bull}{#1}}}
\newcommand*{\Proj}[1]{\ensuremath{\mathbb{P}{#1}}}
\NewDocumentCommand\ProjCot{O{}m}{\ensuremath{\mathbb{P}T_{#1}^*#2}}
%\NewDocumentCommand\CotRay{O{}m}{\ensuremath{ST^*_{#1}#2}}
\newcommand*{\OO}[1]{\ensuremath{\mathcal{O}\left(#1\right)}}
\newcommand*{\Gr}[2]{\ensuremath{\operatorname{Gr}_{#1}\!{#2}}}
\newcommand*{\gr}[1]{\ensuremath{\operatorname{gr} #1}}
\newcommand*{\id}{\ensuremath{\operatorname{id}}}
\newcommand*{\Div}[1]{\ensuremath{\operatorname{div}\of{#1}}}
%\newcommand*{\frameBundle}[1]{\ensuremath{F{#1}}}
%\newcommand*{\frameBundleFiber}[2]{\ensuremath{F_{#1}{#2}}}
%\newcommand*{\orientedFrameBundle}[1]{\ensuremath{F^+{#1}}}
%\newcommand*{\orientedFrameBundleFiber}[2]{\ensuremath{F^+_{#1}{#2}}}
%\newcommand*{\adaptedFrameBundle}[2]{\ensuremath{F_{#1}{#2}}}
%\newcommand*{\geodesicVectorField}{\ensuremath{\mathscr{X}}}
\newcommand*{\graph}[1]{\ensuremath{\operatorname{graph}\of{#1}}}
% Principal value integral
\newcommand*{\pv}{\mathop{\mathrm PV}\!}
\newcommand*{\RP}[1]{\ensuremath{\mathbb{RP}^{#1}}}
\newcommand*{\CP}[1]{\ensuremath{\mathbb{CP}^{#1}}}
\newcommand*{\HP}[1]{\ensuremath{\mathbb{HP}^{#1}}}
\newcommand*{\Sym}[2]{\ensuremath{\operatorname{Sym}^{#1}\of{#2}}}
\NewDocumentCommand\Lm{smm}{\IfBooleanTF{#1}{\ensuremath{\Lambda^{#2}\pr{#3}}}{\ensuremath{\Lambda^{#2}#3}}}
\NewDocumentCommand\Lmtop{sm}%
{\ensuremath{%%
	\IfBooleanTF{#1}%
	{%
		\Lm*{\operatorname{top}}{#2}%
	}%
	{%
		\Lm{\operatorname{top}}{#2}%
	}%
}}%%
\DeclareMathOperator{\indx}{index}
\NewDocumentCommand\opsymbol{om}%
{%%
\IfValueTF{#1}{\sigma_{#1}\of{#2}}{\sigma_{#2}}%
}%%

\NewDocumentCommand\charvariety{om}%
{%%
\ensuremath{%%%
\IfValueTF{#1}{\Xi_{#1,#2}}{\Xi_{#2}}%
}%%%
}%%


\NewDocumentCommand\complexcharvariety{om}%
{%%
\ensuremath{\charvariety[#1]{#2}^{\mathbb{C}}}
}%%

\NewDocumentCommand\longExactSequence{mmm}%
%%% Use:
%%% % First define something like:
%%% \NewDocumentCommand\HA{m}{\cohomology{#1}{X}}
%%% \NewDocumentCommand\HB{m}{\cohomology{#1}{Y}}
%%% \NewDocumentCommand\HC{m}{\cohomology{#1}{Z}}
%%% % and then call
%%% \longExactSequence{\HA}{\HB}{\HC}
{%%%
\begin{center}
\begin{tikzcd}[ampersand replacement=\&]
0 \rar \& #1{0} \rar \& #2{0} \rar \& #3{0} \ar[out=-30, in=150]{dll} \\
{} \& #1{1} \rar \& #2{1} \rar \& #3{1} \ar[out=-30, in=150]{dll} \\
{} \& #1{2} \rar \& #2{2} \rar \& #3{2} \ar[out=-30, in=150]{dll}\\
{} \& #1{3} \rar \& #2{3} \rar \& #3{3} \ar[out=-30, in=150]{dll}\\
{} \& #1{4} \rar \& \cdots\cdots \  \&
\end{tikzcd}
\end{center}
}%%%
\NewDocumentCommand\indicator{m}{\ensuremath{1_{#1}}}
\NewDocumentCommand\DeltaCommutativeDiagram{mmm}% {X}{Y}{Z} gives
%% X ---> Z
%%   \ /
%%    Y
{%
\[
\begin{tikzcd}[column sep=small, row sep=small, ampersand replacement=\&]
{#1} \arrow[rr] \arrow[dr] \& \& {#3} \arrow[dl] \\
{} \& {#2} \& {}
\end{tikzcd}
\]
}%

% outer measure
\NewDocumentCommand\outermeasure{som}%[measure space]{set}, *=parenthesize
{%
\IfValueTF{#2}{%%%%
\IfBooleanTF{#1}%
{m_{#2}\of{#3}}%
{m_{#2}#3}%
}%%%%
{%%%%
\IfBooleanTF{#1}%
{m\of{#3}}%
{m#3}%
}%%%%
}%

% measure
\NewDocumentCommand\measure{som}%[measure space]{set}, *=parenthesize
{%
\IfValueTF{#2}{%%%%
\IfBooleanTF{#1}%
{m_{#2}\of{#3}}%
{m_{#2}#3}%
}%%%%
{%%%%
\IfBooleanTF{#1}%
{m\of{#3}}%
{m#3}%
}%%%%
}%


% L^p space
\NewDocumentCommand\Leb{om}{\ensuremath{\IfValueTF{#1}{L^{#1}\!\pr{#2}}{L^{#2}}}}

%L^p norm
\makeatletter
\NewDocumentCommand\Lnorm{omm}
{%
\ensuremath{\norm{#3}_{\Leb[#1]{#2}}}%
}%
\makeatother


\NewDocumentCommand\homotopyGroup{mm}{\ensuremath{\pi_{#1}\of{#2}}}
\NewDocumentCommand\fundamentalGroup{m}{\homotopyGroup{1}{#1}}
\NewDocumentCommand\lb{smm}{\ensuremath{\left[#2\IfBooleanTF{#1}{,}{}#3\right]}}
\NewDocumentCommand\ChernChar{mm}{\ensuremath{\operatorname{ch}_{#1}\of{#2}}}

\NewDocumentEnvironment{tableau}{}%%
{%%
\left( \ \begin{matrix}
}%%
{%%
\end{matrix} \ \right)
}%%
\NewDocumentCommand\highlightcell{m}{\cellcolor{blue!30}{#1}}
\NewDocumentCommand\freeDeriv{m}{\highlightcell{#1}}
\NewDocumentCommand\centerofmass{m}{\ensuremath{\left<#1\right>}}

\NewDocumentCommand\crossratio{mmmm}{\ensuremath{\pr{#1,#2;#3,#4}}}

\NewDocumentCommand\unitTangentBundle{m}{\ensuremath{UT#1}}
\NewDocumentCommand\normalBundle{m}{\ensuremath{\nu#1}}
\NewDocumentCommand\unitNormalBundle{m}{\ensuremath{U\nu#1}}


% The Moebius strip
% one third of the Moebius Strip
%: \strip{<angle>}
\newcommand{\strip}[1]{%
\shadedraw[draw=black!30,top color=white,bottom color=gray,rotate=#1]
 (0:2.8453) ++ (-30:1.5359) arc (60:0:2)
 -- ++  (90:5) arc (0:60:2) -- ++ (150:3) arc (60:120:2) 
 -- ++ (210:5) arc (120:60:2) -- cycle;}
%: \MoebiusStrip{<text1>}{<text2>}{<text3>}
\newcommand{\MoebiusStrip}[3]{%
\begin{scope} [transform shape]
	\strip{0}
	\strip{120}
	\strip{-120}
	\draw (-60:3.5) node[scale=6,rotate=30] {#1};
	\draw (180:3.5) node[scale=4,rotate=-90]{#3};
	% redraw the first strip after clipping
	\clip (-1.4,2.4)--(-.3,6.1)--(1.3,6.1)--(5.3,3.7)--(5.3,-2.7)--cycle;
	\strip{0}
	\draw (60:3.5) node [gray,xscale=-4,yscale=4,rotate=30]{#2};
\end{scope}}


\newcommand{\transpose}[1]{\tensor[^t]{#1}{}}

% For Lie groups

\NewDocumentCommand\Aff{om}{\ensuremath{\IfValueTF{#1}{\operatorname{Aff}\of{#1}}{\operatorname{Aff}_{#2}}}}
\NewDocumentCommand\LieAff{om}{\ensuremath{\IfValueTF{#1}{\mathfrak{aff}\of{#1}}{\mathfrak{aff}_{#2}}}}
\newcommand{\weirdplus}{\dotplus}

% Backwards vector symbol.
\makeatletter
\NewDocumentCommand\cev{sm}%
{%
\IfBooleanTF{#1}{\overleftarrow{#2}}{\mathpalette\do@cev{#2}}%
}%
\newcommand{\do@cev}[2]{%
  \fix@cev{#1}{+}%
  \reflectbox{$\m@th#1\vec{\reflectbox{$\fix@cev{#1}{-}\m@th#1#2\fix@cev{#1}{+}$}}$}%
  \fix@cev{#1}{-}%
}
\newcommand{\fix@cev}[2]{%
  \ifx#1\displaystyle
    \mkern#23mu
  \else
    \ifx#1\textstyle
      \mkern#23mu
    \else
      \ifx#1\scriptstyle
        \mkern#22mu
      \else
        \mkern#22mu
      \fi
    \fi
  \fi
}

\makeatother

\NewDocumentCommand\reverseLongExactSequence{mmm}%
%%% Use:
%%% % First define something like:
%%% \NewDocumentCommand\HA{m}{\cohomology{#1}{X}}
%%% \NewDocumentCommand\HB{m}{\cohomology{#1}{Y}}
%%% \NewDocumentCommand\HC{m}{\cohomology{#1}{Z}}
%%% % and then call
%%% \longExactSequence{\HA}{\HB}{\HC}
{%%%
\begin{center}
\begin{tikzcd}[ampersand replacement=\&]
       \&            \&            \& \cdots \ar[out=-30, in=150]{dll} \\
    {} \& #1{4} \rar \& #2{4} \rar \& #3{4} \ar[out=-30, in=150]{dll} \\
    {} \& #1{3} \rar \& #2{3} \rar \& #3{3} \ar[out=-30, in=150]{dll} \\
    {} \& #1{2} \rar \& #2{2} \rar \& #3{2} \ar[out=-30, in=150]{dll}\\
    {} \& #1{1} \rar \& #2{1} \rar \& #3{1} \ar[out=-30, in=150]{dll}\\
    {} \& #1{0} \rar \& #2{0} \rar \& #3{0} \ar{r} \& 0\\
\end{tikzcd}
\end{center}
}%%%
